\chapter{Heat and Temperature}
    \section{What is Heat?}
    We could think of heat as:
    \begin{itemize}
        \item The transfer of thermal energy from a region of high thermal
        energy to a region of low thermal energy, provided that the two regions
        are in contact, is known as \textbf{heat}.
        \item The transfer of thermal energy between molecules within a system
    \end{itemize}
    The SI unit for Heat is the joule \unit{J}
    \begin{tcolorbox}[colback=yellow!30!white,
                      colframe=yellow!70!black,
                      title={Just a heads up...}]
        A body can only lose or gain heat but not possess heat as heat is a
        measure of thermal energy and not a property of matter.
    \end{tcolorbox}
    \section{What is Temperature?}
    We can define temperature as:
    \begin{itemize}
        \item (\textbf{\textit{layman}}) the hotness or coldness of a body.
        \item (\textbf{\textit{more technical}}) a measure of the
        \underline{mean} kinetic energy of a body in a thermal energy system.
    \end{itemize}
    The SI unit for Temperature is the kelvin \unit{K}.
    \begin{tcolorbox}[colback=yellow!30!white,
                      colframe=yellow!70!black,
                      title={Just a heads up...}]
        There are many ways in which Temperature is represented. These are:
        \begin{itemize}
            \item \textbf{Celsius} $^{\circ}$\unit{C}: Used in Commonwealth
                Countries and it is also known as the Centigrade scale. It is
                based on the percent division for the range defined by the
                melting and boiling point of water.
            \item \textbf{Fahrenheit} $^{\circ}$\unit{F}: Used in the US and
                Carribean countries. It was defined by German scientist, Daniel
                Gabriel Fahrenheit. In this scale, water freezes at
                32 $^{\circ}$\unit{F}  (0 $^{\circ}$\unit{C}) and
                boils at 212 $^{\circ}$\unit{F}
                (100 $^{\circ}$\unit{C}). The scale allows for
                temperatures below zero and the coldest possible
                tempearature capped at -459.67 $^{\circ}$\unit{F}.
            \item \textbf{Rankine} $^{\circ}$\unit{R}: Not widely used.
                Defined by Scottish scientist William John Rankine. It is to
                Fahrenheit as Celsius as it is to Kelvin. In this scale, water
                freezes at 491.67 $^{\circ}$\unit{R} and boils at
                671.67 $^{\circ}$\unit{R}.
        \end{itemize}

        You can switch between these Celsius and Fahrenheit by the following
        equation:
        \begin{equation}
            C=\frac{9}{5}\left(F-32\right)
            \label{eq:celsius_conversion}
        \end{equation}
    \end{tcolorbox}
